\documentclass[10pt]{article}
\title{The Ultimate TikZ Cheatsheet}
\author{Christian Amstutz}
\date{\today}

\usepackage[landscape, margin=0.5in]{geometry}
\usepackage{multicol}
\usepackage[T1]{fontenc}
\usepackage{keystroke}

\newcommand{\tikzcmd}[1]{\textbf{#1}}
\newcommand{\tikzparam}[1]{\textbf{\emph{#1}}}
\newcommand{\tikzspacer}{\vspace{9pt}}
\newcommand{\tikzlib}[1]{\textbf{#1}}



\begin{document}

    \maketitle

    \begin{multicols}{2}

        \section{Coordinates}
        \tikzcmd{\textbackslash coordinate (\tikzparam{label}) at \tikzparam{coord};} define named coordinate\\
        \emph{Where is (0,0)}?\\
        \emph{Difference between - - and to?}
        \tikzcmd{(\tikzparam{x},\tikzparam{y})} cartesian coordinates\\
        \tikzcmd{(\tikzparam{$\phi$},\tikzparam{r})} polar coordinates ($\phi$ in degrees)\\
        \tikzcmd{+(\tikzparam{coord})} relative coordinate to last coordinate\\
        \tikzcmd{++(\tikzparam{coord})} relative coordinate and move cursor\\
        \tikzcmd{(\tikzparam{coord} -| 1)} horizontal perpendicular to to \tikzparam{coord}\\
        \tikzcmd{(\$(\tikzparam{coord})+(\tikzparam{rel\_coord})\$)} calculate new coordinate \tikzlib{calc}\\
        \tikzcmd{(\$(\tikzparam{A})!0.33!(\tikzparam{B})\$)} coordinate 1/3 on line between A and B \tikzlib{calc}\\
        \tikzcmd{(\$(\tikzparam{A})!\tikzparam{C}!(\tikzparam{B})\$)} projection of C to line between A and B \tikzlib{calc}\\
        \tikzcmd{(\tikzparam{node}.\tikzparam{anchor})} anchor of a node

        \section{Paths}
        \tikzcmd{\textbackslash path[keys] (coord1) object (coord2);}\\
        keys:\\
        \tikzcmd{draw} draw the path by line\\
        \tikzcmd{draw=none} do not draw line\\
        \tikzcmd{solid}/\tikzcmd{dashed}/\tikzcmd{dotted} line dashing\\
        \tikzcmd{ultra thin}/\tikzcmd{normal}/\tikzcmd{ultra tick} line width\\
        \tikzcmd{fill=\tikzparam{color}} fills the object with \tikzparam{color}\\
        \tikzcmd{color=\tikzparam{color}} sets the line color\\
        objects:\\
        \tikzcmd{- -} draw line\\
        \tikzcmd{->}/\tikzcmd{<-}/\tikzcmd{<->} draw arrow\\
        \tikzcmd{rectangle} draw rectangle between coordinates\\
        \tikzcmd{circle} draw circle\\
        \tikzcmd{ellipse}\\
        \tikzcmd{node} places a node at current postion\\
        \tikzcmd{plot}\\
        line control:\\
        \tikzcmd{in} / \tikzcmd{out}\\
        \tikzcmd{control points}\\
        \tikzcmd{intersections}\\
        \tikzcmd{nodes at lines}

        \section{Nodes}
        \tikzcmd{\textbackslash node[keys] (label) at (coord) \{content\};}\\
        \tikzcmd{\textbackslash begin\{minipage\}\{\tikzparam{content}\}\textbackslash end\{minipage\} } multi-line content\\
        keys:\\
        the keys of a path can be also used here\\
        \tikzcmd{inner sep=\tikzparam{dist}} distance from content to node\\
        \tikzcmd{outer sep=\tikzparam{dist}} distance from node to other objects\\
        \tikzcmd{min width=\tikzparam{dist}} minimum width of node\\
        \tikzcmd{min height=\tikzparam{dist}} minimum height of node\\
        \tikzcmd{xshift=\tikzparam{dist}} / \tikzcmd{yshift=\tikzparam{dist}} shifting the node by \tikzparam{dist}\\
        \tikzcmd{shift=\{(rel\_coord)\}} shift node\\
        \tikzcmd{font=\tikzparam{fontconfig}} TeX commands to format text in node\\
        \tikzcmd{node seperation=\tikzparam{dist}}

        \section{Pics}
        \tikzcmd{\textbackslash tikzset\{\tikzparam{pic name}/.pic=\{\tikzparam{pic definition}\}\}} define new pic (short form)\\
        \tikzcmd{\textbackslash pic[\tikzparam{keys}] (\tikzparam{label}) \{\tikzparam{pic\_type}\} }
        \begin{itemize}

            \item arguments
            \item accessing hierarchy
        \end{itemize}

        \section{Keys, styles, etc.}
        %\tikzstyle{every label}=[\tikzparam{keys}] --- deprecated\\
        %\tikzstyle{my style}[red]=[draw=#1,fill=#1!20]
        \tikzcmd{\textbackslash tikzset\{\tikzparam{mystyle}/.style=\{\tikzparam{keys}\}\}} define a new style\\
        \tikzcmd{\textbackslash tikzset\{\tikzparam{mystyle}/.style=\{label=\#1\}\}} a parameter \tikzparam{\#1} can be used\\

        %\begin{tikzpicture}
        %[some options,
        %first style/.style={blah},
        %second style/.style={blub}]    ...
        %\end{tikzpicture}

        \tikzcmd{\textbackslash pgfkeys\{/key family/key/.subkey = \{\tikzparam{value list}\}\}} basic operation to set PGF keys

        \section{Layers}

    \end{multicols}

\end{document}
