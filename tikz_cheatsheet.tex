\documentclass[10pt]{article}
\title{TikZ Cheatsheet}
\author{Christian Amstutz}
\date{\today}

\usepackage[landscape, margin=0.5in]{geometry}
\usepackage{multicol}
\usepackage[T1]{fontenc}
\usepackage{keystroke}

\newcommand{\tikzcmd}[1]{\textbf{#1}}
\newcommand{\tikzparam}[1]{\textbf{\emph{#1}}}
\newcommand{\tikzspacer}{\vspace{9pt}}



\begin{document}

    \maketitle

    \begin{multicols}{2}

        \section{Coordinates}
        \emph{Where is (0,0)}?\\
        \tikzcmd{(x,y)} cartesian coordinates\\
        \tikzcmd{($\phi$,r)} polar coordinates ($\phi$ in degrees)\\
        \tikzcmd{+(coord)} relative coordinate to last coordinate\\
        \tikzcmd{++(coord)} relative coordinate and move cursor\\
        \tikzcmd{(coord1 -| 1)} perpendicular to coord1\\
        \tikzcmd{(\$(coord1)+(rel\_coord)\$)} calculate new coordinate \\
        \tikzcmd{(node.anchor)} anchor of a node\\

    \end{multicols}

\end{document}
